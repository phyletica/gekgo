\begin{frame}
    \frametitle{Key findings}
    \begin{itemize}
        \item<1-> Strong support that all 8 pairs of \spp{Cyrtodactylus}
            populations diverged independently
        \item<2-> Weak support that all 8 pairs of \spp{Gekko} populations
            diverged independently
        \item<3-> Simulation results suggest \ecoevolity can accurately
            estimate the timing and number of divergences given the gekkonid
            RADseq data
    \end{itemize}
\end{frame}

\begin{frame}
    \frametitle{Caveats}
    \begin{itemize}
        \item Too few island pairs to rule out climate-driven vicariant
            speciation
        \item Differences in divergence times could be due to variation in
            fragmentation times among island pairs
        \item Differences in divergence could also be due to variation
            in mutation rates
    \end{itemize}

    \vspace{1cm}
    \begin{itemize}
        \item<2-> Seems safe to conclude that the ``species-pump'' is not the
            rule for gekkonids, but maybe the exception
    \end{itemize}
\end{frame}

\begin{frame}
    \frametitle{Take home points}
    \begin{itemize}
        \item<1-> Support against the ``species-pump'' hypothesis
        \item<2-> Repeated cycles of climate-driven island fragmentation was
            not an the primary mechanism of speciation for gekkonid lizards in
            the Philippines
        \item<3-> Rare over-water dispersal via rafting on vegetation
            is likely the primary mechanism responsible for the distribution
            of gekkonid lizards in the Philippines
    \end{itemize}
\end{frame}
