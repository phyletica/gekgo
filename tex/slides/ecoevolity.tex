\begin{frame}

    Analyzed RADseq data with full-likelihood Bayesian comparative phylogeographic method:

    \begin{center}
        \LARGE
        \href{https://github.com/phyletica/ecoevolity}{
            \textbf{\textcolor{pgreen}{E}\textcolor{pteal}{co\textcolor{pauburn}{evo}lity}}}:
        \textcolor{pgreen}{\bf E}stimating \textcolor{pauburn}{\bf evo}lutionary \textcolor{pteal}{\bf coevality}
    \end{center}

    % \begin{itemize}
    %     \item<2-> CTMC model of characters evolving along genealogies
    %     \item<2-> Coalescent model of genealogies branching within populations
    %     \item<2-> Dirichlet-process prior across divergence models
    %     \item<2-> Gibbs sampling\footnote{\tiny\shortfullcite{Neal2000}}
    %               to numerically sample models
    %     \item<2-> Analytically integrate over genealogies\footnote{\tiny\shortfullcite{Bryant2012}}

    %     \bigskip
    %     \item<3-> \textsl{Goal: Fast, full-likelihood Bayesian method to infer
    %             patterns of co-diversification from genome-scale data}
    % \end{itemize}
    \begin{itemize}
        \item<2-> Used simulations to assess how well \ecoevolity works given
            the gekkonid RADseq \datasets
    \end{itemize}
\end{frame}
