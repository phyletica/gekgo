\siFigure{../../../data/genomes/msg/ecoevolity-results/grid-cyrtodactylus-sumevents.pdf}{
    Approximate prior (light bars) and posterior (dark bars) probabilities
    of numbers of divergence events across pairs of \spp{Cyrtodactylus}
    populations
    under three different priors on the concentration parameter of
    the Dirichlet process.
    Bayes factors for each number of divergence times is given above the
    corresponding bars.
    Each Bayes factor compares the corresponding number of events 
    to all other possible numbers of divergence events.
    \weusedggplot
}{fig:cyrtneventsbyconcentration}

\siFigure{../../../data/genomes/msg/ecoevolity-results/grid-cyrtodactylus-sumtimes.pdf}{
    Approximate marginal posterior densities of divergence times for each
    pair of \spp{Cyrtodactylus} populations
    under three different priors on the concentration parameter of
    the Dirichlet process.
    \weusedggridges
}{fig:cyrtdivtimesbyconcentration}

\siFigure{../../../data/genomes/msg/ecoevolity-results/grid-cyrtodactylus-sumevents-nopoly.pdf}{
    Approximate prior (light bars) and posterior (dark bars) probabilities
    of numbers of divergence events across pairs of \spp{Cyrtodactylus}
    populations
    under four different combinations of prior on divergence times (rows)
    and recoding or removing polyallelic characters (columns).
    Bayes factors for each number of divergence times is given above the
    corresponding bars.
    Each Bayes factor compares the corresponding number of events 
    to all other possible numbers of divergence events.
    \weusedggplot
}{fig:cyrtneventsbytimeprior}

\siFigure{../../../data/genomes/msg/ecoevolity-results/grid-cyrtodactylus-sumtimes-nopoly.pdf}{
    Approximate marginal posterior densities of divergence times for each
    pair of \spp{Cyrtodactylus} populations
    under four different combinations of prior on divergence times (rows)
    and recoding or removing polyallelic characters (columns).
    \weusedggridges
}{fig:cyrtdivtimesbytimeprior}



\siFigure{../../../data/genomes/msg/ecoevolity-results/grid-cyrtodactylus-sumsizes.pdf}{
    Approximate marginal posterior densities of population sizes for each
    pair of \spp{Cyrtodactylus} populations
    under three different priors on the concentration parameter of
    the Dirichlet process.
    \weusedggridges
}{fig:cyrtpopsizesbyconcentration}

\siFigure{../../../data/genomes/msg/ecoevolity-results/grid-cyrtodactylus-sumsizes-nopoly.pdf}{
    Approximate marginal posterior densities of population sizes for each
    pair of \spp{Cyrtodactylus} populations
    under four different combinations of prior on divergence times (rows)
    and recoding or removing polyallelic characters (columns).
    \weusedggridges
}{fig:cyrtpopsizesbytimeprior}



\siFigure{../../../data/genomes/msg/ecoevolity-results/grid-gekko-sumevents.pdf}{
    Approximate prior (light bars) and posterior (dark bars) probabilities
    of numbers of divergence events across pairs of \spp{Gekko}
    populations
    under three different priors on the concentration parameter of
    the Dirichlet process.
    Bayes factors for each number of divergence times is given above the
    corresponding bars.
    Each Bayes factor compares the corresponding number of events 
    to all other possible numbers of divergence events.
    \weusedggplot
}{fig:gekkoneventsbyconcentration}

\siFigure{../../../data/genomes/msg/ecoevolity-results/grid-gekko-sumtimes.pdf}{
    Approximate marginal posterior densities of divergence times for each
    pair of \spp{Gekko} populations
    under three different priors on the concentration parameter of
    the Dirichlet process.
    \weusedggridges
}{fig:gekkodivtimesbyconcentration}

\siFigure{../../../data/genomes/msg/ecoevolity-results/grid-gekko-sumevents-nopoly.pdf}{
    Approximate prior (light bars) and posterior (dark bars) probabilities
    of numbers of divergence events across pairs of \spp{Gekko}
    populations
    under six different combinations of prior on divergence times (rows)
    and recoding or removing polyallelic characters (columns).
    Bayes factors for each number of divergence times is given above the
    corresponding bars.
    Each Bayes factor compares the corresponding number of events 
    to all other possible numbers of divergence events.
    \weusedggplot
}{fig:gekkoneventsbytimeprior}

\siFigure{../../../data/genomes/msg/ecoevolity-results/grid-gekko-sumtimes-nopoly.pdf}{
    Approximate marginal posterior densities of divergence times for each
    pair of \spp{Gekko} populations
    under six different combinations of prior on divergence times (rows)
    and recoding or removing polyallelic characters (columns).
    \weusedggridges
}{fig:gekkodivtimesbytimeprior}


\siFigure{../../../data/genomes/msg/ecoevolity-results/grid-gekko-sumsizes.pdf}{
    Approximate marginal posterior densities of population sizes for each
    pair of \spp{Gekko} populations
    under three different priors on the concentration parameter of
    the Dirichlet process.
    \weusedggridges
}{fig:gekkopopsizesbyconcentration}

\siFigure{../../../data/genomes/msg/ecoevolity-results/grid-gekko-sumsizes-nopoly.pdf}{
    Approximate marginal posterior densities of population sizes for each
    pair of \spp{Gekko} populations
    under six different combinations of prior on divergence times (rows)
    and recoding or removing polyallelic characters (columns).
    \weusedggridges
}{fig:gekkopopsizesbytimeprior}

\siFigure{../../../data/genomes/msg/ecoevolity-simulations/plots/ancestor-size-scatter.pdf}{
    \jroeditnote{New figure.}
    \jroedit{}{
    The accuracy and precision of \ecoevolity estimates of the ancestral
    population size (scaled by the mutation rate) when we simulated data to
    match our \spp{Cyrtodactylus} (left) and \spp{Gekko} (right) RADseq
    \datasets when all sites (top) or only one SNP per locus (bottom) is
    sampled and analyzed.
    Each circle and associated error bars represent the posterior mean
    and 95\% credible interval.
    Estimates for which the potential-scale reduction factor
    % Details of the PSRF calculation are in the main text, so we
    % don't need this
    % \citep[PSRF; the square root of Equation 1.1 in][]{Brooks1998}
    was greater than 1.2
    \citep{Brooks1998} are highlighted in orange.
    Each plot consists of 4000 estimates---500 simulated \datasets, each with
    eight pairs of populations.
    For each plot, the root-mean-square error (RMSE) and the proportion of
    estimates for which the 95\% credible interval contained the true
    value---$p(\epopsize{}\murate{} \in \textrm{CI})$---is given.
    \weusedmatplotlib
    }
}{fig:simsancestralsizes}

\siFigure{../../../data/genomes/msg/ecoevolity-simulations/plots/descendant-size-scatter.pdf}{
    \jroeditnote{New figure.}
    \jroedit{}{
    The accuracy and precision of \ecoevolity estimates of the descendant
    population sizes (scaled by the mutation rate) when we simulated data to
    match our \spp{Cyrtodactylus} (left) and \spp{Gekko} (right) RADseq
    \datasets when all sites (top) or only one SNP per locus (bottom) is
    sampled and analyzed.
    Each circle and associated error bars represent the posterior mean
    and 95\% credible interval.
    Estimates for which the potential-scale reduction factor
    % Details of the PSRF calculation are in the main text, so we
    % don't need this
    % \citep[PSRF; the square root of Equation 1.1 in][]{Brooks1998}
    was greater than 1.2
    \citep{Brooks1998} are highlighted in orange.
    Each plot consists of 8000 estimates---500 simulated \datasets, each with
    eight pairs of populations.
    For each plot, the root-mean-square error (RMSE) and the proportion of
    estimates for which the 95\% credible interval contained the true
    value---$p(\epopsize{}\murate{} \in \textrm{CI})$---is given.
    \weusedmatplotlib
    }
}{fig:simsdescendantsizes}

\siFigure{../../../data/genomes/msg/ecoevolity-simulations/plots/cyrt-event-time-sampling-disparity-scatter.pdf}{
    \jroeditnote{New figure.}
    \jroedit{}{
    The accuracy and precision of \ecoevolity divergence-time estimates (in
    units of expected subsitutions per site) from data simulated to match our
    pairs of \spp{Cyrtodactylus philippinicus} populations from the Islands of
    (left) Luzon and Babuyan Claro and (right) Polillo and Luzon. 
    The number of individuals sampled from each island population is indicated
    in parantheses at the top of each column of plots.
    Results are shown when all sites (top) or only one SNP per locus (bottom)
    is sampled and analyzed.
    These are a subset of the data plotted in Figure~\ref{fig:simsdivtimes};
    they are plotted separately here to compare two pairs of populations with
    large differences in the number of sampled individuals and loci.
    Each circle and associated error bars represent the posterior mean and 95\%
    credible interval for the time that a pair of populations diverged.
    Estimates for which the potential-scale reduction factor
    % Details of the PSRF calculation are in the main text, so we
    % don't need this
    % \citep[PSRF; the square root of Equation 1.1 in][]{Brooks1998}
    was greater than 1.2
    \citep{Brooks1998} are highlighted in orange.
    Each plot consists of 500 estimates---one from each of the 500 simulated
    \datasets.
    For each plot, the root-mean-square error (RMSE) and the proportion of
    estimates for which the 95\% credible interval contained the true
    value---$p(\comparisondivtime{} \in \textrm{CI})$---is given.
    \weusedmatplotlib
    }
}{fig:sampledisparitydivtimes}
