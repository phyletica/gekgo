A primary goal of biogeography is to understand how large-scale environmental
processes, like climate change, affect diversification.
Over the past three million years, the landscape of the Philippine Islands has
repeatedly coalesced and fragmented due to sea-level changes associated with
the glacial cycles.
This repeated climate-driven vicariance has been proposed as a model of
speciation across the islands.
This model predicts speciation times that are temporally clustered around
the times when interglacial rises in sea levels fragmented the islands.
We collect comparative genomic data from 16 pairs of insular gecko populations
to test this prediction.
Specifically, we analyze these data in a full-likelihood, Bayesion model-choice
framework to test for shared divergence times among the pairs.
Our results support that each pair of gecko populations diverged independently.
These results suggest the repeated bouts of climate-driven landscape
fragmentation has not been an important mechanism of speciation for
gekkonid lizards on the Philippine Islands.
