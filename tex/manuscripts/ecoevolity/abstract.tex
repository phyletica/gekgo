A primary goal of biogeography is to understand how large-scale environmental
processes, like climate change, affect diversification.
One often-invoked but seldom tested process is the \jroedit{so-called}{} ``species-pump''
model, in which repeated bouts of co-speciation is driven by oscillating
climate-induced habitat connectivity cycles.
For example, over the past three million years, the landscape of the Philippine
Islands has repeatedly coalesced and fragmented due to sea-level changes
associated with \jroedit{the}{} glacial cycles.
This repeated climate-driven vicariance has been proposed as a model of
speciation across evolutionary lineages codistributed throughout the islands.
This model predicts speciation times that are temporally clustered around
the times when interglacial rises in sea level fragmented the islands.
\jroedit{%
Given the significance and conceptual impact the model has shown, surprisingly
few tests of this prediction have been provided.}{}
\jroedit{}{%
To test this prediction,}
\jroedit{We}{we} collected comparative genomic data from 16 pairs of insular gecko populations.
\jroedit{to test the prediction of temporally clustered divergences.}{}
\jroedit{Specifically, we}{We} analyze these data in a full-likelihood, Bayesian model-choice
framework to test for shared divergence times among the pairs.
Our results provide support against the species-pump model prediction in favor
of an alternative interpretation, namely that each pair of gecko populations
diverged independently.
These results suggest the repeated bouts of climate-driven landscape
fragmentation has not been an important mechanism of speciation for
gekkonid lizards on the Philippine Islands.
\jroedit{%
Interpretations of shared mechanisms of diversification historically have been
pervasive in biogeography, often advanced on the basis of taxonomy-based
depictions of species distributions.
Our results call for possible re-evaluation of other, classic
co-diversification studies in a variety of geographic systems.}{}
