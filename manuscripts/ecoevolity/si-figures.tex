\siFigure{../../data/genomes/msg/ecoevolity-results/pyco-sumsizes-cyrtodactylus-rate200-pretty-pycoevolity-sizes.pdf}{
    The approximate marginal posterior densities of effective population size
    for each \spp{Cyrtodactylus} population.
}{fig:cyrtpopsizes}

\siFigure{../../data/genomes/msg/ecoevolity-results/pyco-sumsizes-gekko-rate2000-pretty-pycoevolity-sizes.pdf}{
    The approximate marginal posterior densities of effective population size
    for each \spp{Gekko} population.
}{fig:gekkopopsizes}

\siFigure{../../data/genomes/msg/ecoevolity-results/grid-cyrtodactylus-sumtimes.pdf}{
    The effect on \spp{Cyrtodactylus} divergence time estimates by three
    different \emph{a priori} assumptions about the concentration parameter
    (\concentration) of the Dirichlet process prior on divergence models.
    The plots show the approximated marginal posterior densities of divergence
    times for each pair of populations.
}{fig:cyrtdivtimesbyconcentration}

\siFigure{../../data/genomes/msg/ecoevolity-results/grid-cyrtodactylus-sumtimes-nopoly.pdf}{
    The effect on \spp{Cyrtodactylus} divergence time estimates by the prior
    assumption about divergence times (rows), and whether characters with three
    or more states (polyallelic characters) are removed or recoded as binary
    (columns).
    The plots show the approximated marginal posterior densities of divergence
    times for each pair of populations.
}{fig:cyrtdivtimesbytimeprior}

\siFigure{../../data/genomes/msg/ecoevolity-results/grid-cyrtodactylus-sumevents.pdf}{
    The effect on the inferred posterior probabilities of the number of
    divergence events across \spp{Cyrtodactylus} pairs by three different
    assumptions about the concentration parameter (\concentration) of the
    Dirichlet process prior on divergence models.
    Light and dark bars show the prior and approximated posterior
    probabilities, respectively.
    The Bayes factor for each number of divergence times is above the bars.
    Each Bayes factor compares the given number of events ($p(\nevents = i)$)
    to all other possible numbers of divergence events ($p(\nevents \neq i)$).
}{fig:cyrtneventsbyconcentration}

\siFigure{../../data/genomes/msg/ecoevolity-results/grid-cyrtodactylus-sumevents-nopoly.pdf}{
    The effect on the inferred posterior probabilities of the number of
    divergence events across \spp{Cyrtodactylus} pairs by the prior assumption
    about divergence times (rows), and whether characters with three or more
    states (polyallelic characters) are removed or recoded as binary (columns).
    Light and dark bars show the prior and approximated posterior
    probabilities, respectively.
    The Bayes factor for each number of divergence time is above the bars.
    Each Bayes factor compares the given number of events ($p(\nevents = i)$)
    to all other possible numbers of divergence events ($p(\nevents \neq i)$.
}{fig:cyrtneventsbytimeprior}

\siFigure{../../data/genomes/msg/ecoevolity-results/grid-cyrtodactylus-sumsizes.pdf}{
    The effect on \spp{Cyrtodactylus} population size estimates by three
    different \emph{a priori} assumptions about the concentration parameter
    (\concentration) of the Dirichlet process prior on divergence models.
    The plots show the approximated marginal posterior densities of effective
    population size (scaled by the mutation rate) for each pair of populations.
}{fig:cyrtpopsizesbyconcentration}

\siFigure{../../data/genomes/msg/ecoevolity-results/grid-cyrtodactylus-sumsizes-nopoly.pdf}{
    The effect on \spp{Cyrtodactylus} population size estimates by the prior
    assumption about divergence times (rows), and whether characters with three
    or more states (polyallelic characters) are removed or recoded as binary
    (columns).
    The plots show the approximated marginal posterior densities of effective
    population size (scaled by the mutation rate) for each pair of populations.
}{fig:cyrtpopsizesbytimeprior}



\siFigure{../../data/genomes/msg/ecoevolity-results/grid-gekko-sumtimes.pdf}{
    The effect on \spp{Gekko} divergence time estimates by three
    different \emph{a priori} assumptions about the concentration parameter
    (\concentration) of the Dirichlet process prior on divergence models.
    The plots show the approximated marginal posterior densities of divergence
    times for each pair of populations.
}{fig:gekkodivtimesbyconcentration}

\siFigure{../../data/genomes/msg/ecoevolity-results/grid-gekko-sumtimes-nopoly.pdf}{
    The effect on \spp{Gekko} divergence time estimates by the prior
    assumption about divergence times (rows), and whether characters with three
    or more states (polyallelic characters) are removed or recoded as binary
    (columns).
    The plots show the approximated marginal posterior densities of divergence
    times for each pair of populations.
}{fig:gekkodivtimesbytimeprior}

\siFigure{../../data/genomes/msg/ecoevolity-results/grid-gekko-sumevents.pdf}{
    The effect on the inferred posterior probabilities of the number of
    divergence events across \spp{Gekko} pairs by three different assumptions
    about the concentration parameter (\concentration) of the Dirichlet process
    prior on divergence models.
    Light and dark bars show the prior and approximated posterior
    probabilities, respectively.
    The Bayes factor for each number of divergence times is above the bars.
    Each Bayes factor compares the given number of events ($p(\nevents = i)$)
    to all other possible numbers of divergence events ($p(\nevents \neq i)$).
}{fig:gekkoneventsbyconcentration}

\siFigure{../../data/genomes/msg/ecoevolity-results/grid-gekko-sumevents-nopoly.pdf}{
    The effect on the inferred posterior probabilities of the number of
    divergence events across \spp{Gekko} pairs by the prior assumption about
    divergence times (rows), and whether characters with three or more states
    (polyallelic characters) are removed or recoded as binary (columns).
    Light and dark bars show the prior and approximated posterior
    probabilities, respectively.
    The Bayes factor for each number of divergence time is above the bars.
    Each Bayes factor compares the given number of events ($p(\nevents = i)$)
    to all other possible numbers of divergence events ($p(\nevents \neq i)$.
}{fig:gekkoneventsbytimeprior}

\siFigure{../../data/genomes/msg/ecoevolity-results/grid-gekko-sumsizes.pdf}{
    The effect on \spp{Gekko} population size estimates by three
    different \emph{a priori} assumptions about the concentration parameter
    (\concentration) of the Dirichlet process prior on divergence models.
    The plots show the approximated marginal posterior densities of effective
    population size (scaled by the mutation rate) for each pair of populations.
}{fig:gekkopopsizesbyconcentration}

\siFigure{../../data/genomes/msg/ecoevolity-results/grid-gekko-sumsizes-nopoly.pdf}{
    The effect on \spp{Gekko} population size estimates by the prior
    assumption about divergence times (rows), and whether characters with three
    or more states (polyallelic characters) are removed or recoded as binary
    (columns).
    The plots show the approximated marginal posterior densities of effective
    population size (scaled by the mutation rate) for each pair of populations.
}{fig:gekkopopsizesbytimeprior}
