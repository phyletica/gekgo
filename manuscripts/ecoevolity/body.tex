\section{Introduction}

An open question in biogeography is to what extent does the landscape, and
changes to it, predict diversification.

Hypothesis: Repeated fragmentation of islands caused vicariant diversification.
This predicts that divergences between populations on separate islands that
were fragmented were temporally clustered and corresponded with interglacial
rises in sea level.

We finally have the data and method for a powerful evaluation of this
hypothesis.


\section{Methods}

\subsection{Sampling}
For two genera of geckos, \spp{Cyrtodactylus} and \spp{Gekko}, we sampled
individuals from pairs of populations that occur on two different islands.
Because the climate-mediated fragmentation of the islands was a relatively
recent phenomenon, we selected pairs of populations that were inferred to be
closely related from previous genetic data
\citep{Siler2012, Siler2014kikuchii, Welton2010, Welton2010zootaxa, Siler2010}.
In other words, we avoided pairs that we knew \emph{a priori} diverged well
before the connectivity cycles, because these cannot provide insight into
whether divergences were clustered \emph{during} these cycles.

We also sought to sample pairs that span islands that were connected during
glacial periods, as well as islands that were never connected.
We included the latter as ``controls.''
Because these islands were never connected, the distribution of the populations
inhabiting them can only be explained by inter-island dispersal.
The divergence between these populations was either due to that
dispersal, or an earlier intra-island divergence.
Either way, the timing of divergences across islands that were never connected
are not expected to be clustered across pairs.
These controls are important given the tendancy of previous methods for this
inference problem that can easily be biased toward over-estimating shared
divergences \citep{Oaks2012,Oaks2014reply}.
Finding shared shared divergence times among pairs for which there is no
tenable mechanism for shared divergences will indicate a problem and prevent us
for misinterpreting shared divergences among pairs spanning islands that were
fragmented as evidence of shared vicariant diversification.
Applying these criteria, for both genera we ended up with eight pairs of
populations, of which three span islands that were never connected, four span
islands that were connected, and one spans islands that were possibly connected
(Table \ref{table:comparisons} \& \highLight{SX}).

\thought{Need more text here about the geology of the ``maybe'' island pairs}

\subsection{Genomic library preparation and sequencing}

We extracted DNA from tissue using the guanidine thiocyanate method described
by Esselstyn et.\ al.\ \citeyear{Esselstyn2008}.
We diluted the extracted DNA for each individual to a concentration of 5
ng/$\mu$L based on the initial concentration measured with a Qubit 2.0
Fluorometer.
We generated three restriction-site associated DNA sequence (RADseq) libraries,
each with 96 individuals, using the multiplexed shotgun genotyping (MSG)
protocol of Andolfatto et.\ al.\ \citep{Andolfatto2011}.
Following digestion of 50 ng of DNA with the NdeI restriction enzyme, we
ligated each sample to one of 96 adaptors with a unique six base-pair barcode.
After pooling the 96 samples together, we selected 250--300bp fragments to
remain in the library using a Pippen Prep.
For each pool of 96 size-selected samples, we performed eight separate
polymerase chain reactions for 14 cycles (PCR) using Phusion High-Fidelity PCR
Master Mix (NEB Biolabs) and primers that bind to common regions in the
adaptors.
Following PCR, we did two rounds of AMPure XP bead cleanup (Beckman Coulter,
Inc.) using a 0.8 bead volume to sample ratio.
Each library was sequenced in one lane of an Illumina Hiseq 2500 high-output
run, with single-end 100bp reads.
We provide information on all of the individuals included in our three RADseq
libraries in \highLight{Table~SX}, a subset of which were included in the
population pairs we analyzed for this study (Table \ref{table:comparisons} \&
\highLight{SX}).

\subsection{Data assembly}
We used ipyrad version 0.7.13 \citep{ipyrad0713} to assemble the raw RADseq
reads into loci.
To maximize the number of assembled loci, we de novo assembled the reads
separately for each pair of populations.
\highLight{Details on ipyrad settings \ldots}

\subsection{Inferring shared divergences}
We approach the inference of temporally clustered divergences as a problem of
model choice.
Our goal is to treat the number of divergence events shared (or not) among the
pairs of populations, and the assignment of the pairs to those events, as
random variables to be estimated from the aligned sequence data.
For eight pairs, there are 4,140 possible divergence models.
I.e., there are 4,140 ways to partition the eight pairs to $\nevents = 1, 2,
\ldots, 8$ divergence events \citep{Bell1934,Oaks2014dpp,Oaks2018ecoevolity}.
Whereas divergences caused by sea-level rise would not happen simultaneously,
we expect that on a timescale of the lizards' mutation rate, treating them as
simultaneous should be a better explanation of the data generated by such a
process than treating them as independent.

Given the large number of models, and our goal of making probability statements
about them, we used a Bayesian model-averaging approach.
Specifically, we used the full-likelihood Bayesian comparative biogeography
method implemented in the software package \ecoevolity version 0.1 (commit
b9f34c8) \citep{Oaks2018ecoevolity}.
This method models each pair of populations as a two-tipped ``species'' or
``population'' tree, with an unknown, constant population size along each of
the three branches, and an unknown time of divergence.
This method can directly estimate the likelihood of values of these unknown
parameters from orthologous biallalic characters by analytically integrating
over all possible gene trees and mutational histories \citep{Bryant2012,
    Oaks2018ecoevolity}.
Within this full-likelihood framework, this method uses a Dirichlet process
prior on the assignment of our pairs to an unknown number of divergence times.
The Dirichlet process is specified by a
(1) concentration parameter, \concentration, which determines how probable it
is for pairs to share the same divergence event, \emph{a priori}, and
(2) base distribution, which serves as the prior on the unique divergence
times.

Importantly, because the pairs of populations are modeled as disconnected
``species'' or ``population'' trees, the relative rates of mutation among the
pairs is not identifiable.
This requires us to make informative prior assumptions about the relative rates
of mutation among the pairs by either constraining them or placing informative
prior probability distributions on them.
Because \spp{Cyrtodactylus} and \spp{Gekko} are deeply divergent
\citationNeeded, and we know very little about their relative rates of
mutation, we analyzed the two genera separately.
Within each genus, the populations are all closely related \citep{Welton2010,
    Welton2010zootaxa, Siler2010, Siler2012, Siler2014kikuchii} allowing us to
make the simplifying assumption that the rates of mutation are equal across
pairs \emph{within} each genus.

Based on previous data \citep{Welton2010, Welton2010zootaxa, Siler2010} we
assumed a prior of \dexponential{0.005} on divergence times for our eight pairs
of \spp{Cyrtodactylus} populations, in units of substitutions per site.
To explore the sensitivity of our results to this assumption, we also
tried a prior of \dexponential{0.05} on the divergence times.
Based on previous data \citep{Siler2012, Siler2014kikuchii}, we assumed a prior
of \dexponential{0.0005} on divergence times for our eight pairs of \spp{Gekko}
populations, in units of substitutions per site.
To explore the sensitivity of our results to this assumption, we also tried
priors of \dexponential{0.005}, \dexponential{0.05} on the \spp{Gekko}
divergence times.

For the concenration parameter of the Dirichlet process, we assumed
a hyperprior of $\distgamma(1.1, 56.1)$ for both genera.
This places approximately half of the prior probability on the model
with no shared divergences ($\nevents = 8$).
By placing most of the prior probability on the independent divergences model,
if we find posterior support for shared divergences, we can be more confident
it is being driven by the data.
To explore the sensitivity of our results to this assumption, we also
tried a hyperprior of
$\distgamma(1.5, 3.13)$
and
$\distgamma(0.5, 1.31)$.
The former corresponds with a prior mean number of divergence events of five,
whereas the latter places 50\% of the prior probability on the single
divergence ($\nevents = 1$) model.

For all analyses of both the \spp{Cyrtodactylus} and \spp{Gekko} data, we
assumed equal mutation rates among the pairs, a prior distribution of
\dgamma{4.0}{0.004} on the effective effective size of the populations, and a
prior distribution of \dgamma{100}{1} on the relative effective size of the
ancestral population (relative to the mean size of the two descendant
populations).
The model implemented in \ecoevolity assumes each character is unlinked (i.e.,
evolved along a gene tree that is independent conditional on the population
tree).
However, by analyzing simulated data, Oaks \citeyear{Oaks2018ecoevolity} showed
the method performs better when all linked sites are used than when data are
excluded to avoid violating the assumption of unlinked sites.
Accordingly, we analyzed all of the sites of our RADseq loci.
The model implemented in \ecoevolity is also restricted to characters with two
possible states (hereafter referred to as polyallelic sites).
Thus, for sites with three or more nucleotides, we compared how sensitive our
results were to two different strategies:
(1) removing polyallelic sites, and
(2) recoding the sites as biallelic by coding each state as either having the
first nucleotide in the alignment or a different nucleotide.
We assumed the biallelic equivalent of a Jukes-Cantor model of character
substitution \citep{JC1969} so that our results are not sensitive to how
nucleotides are coded as binary.

For each analysis, we ran 10 independent MCMC chains for 150,000 generations,
sampling every 100th generation.
We assessed convergence and mixing of the chains by inspecting the potential
scale reduction factor (the square root of Equation 1.1 in Brooks and Gelman
\citeyear{Brooks1998}) and effective sample size \citep{Gong2014} of the log
likelihood and all continuous parameters using the sumchains tool of
\pycoevolity.
We also visually inspected the trace of the log likelihood and parameters over
generations with the program Tracer version 1.6 \citep{Tracer16}.


\section{Results}

Table~\ref{table:comparisons} summarize the number of individuals sampled for
each pair of islands, along with the number of assembled loci, and the number
of total, variable, and polyallelic characters.

The 10 independent MCMC chains of all our \ecoevolity analyses appeared to have
converged almost immediately.
We conservatively removed the first 101 samples, leaving 1400 samples from each
chain (14,000 samples for each analysis).
With the first 101 samples removed, across all our analyses, all ESS values
were greater than 2000, and all PSRF values were less than 1.005.

\subsection{\spp{Cyrtodactylus}}
Support for no shared divergences
(Figures \ref{fig:cyrtdivtimes} \& \ref{fig:cyrtnevents}).
This support is consistent across all three priors on the concentration
parameter of the Dirichlet process
(Figures S\ref{fig:cyrtdivtimesbyconcentration} \& S\ref{fig:cyrtneventsbyconcentration}).
The support is also consistent across both priors on divergence times
and whether polyallelic sites are recoded or removed
(Figures S\ref{fig:cyrtdivtimesbytimeprior} \& S\ref{fig:cyrtneventsbytimeprior}).
Estimates of effective population sizes are very robust to
priors on \concentration and \divtime, and whether polyallelic sites
are recoded or removed
(Figures S\ref{fig:cyrtpopsizesbyconcentration} \& S\ref{fig:cyrtpopsizesbytimeprior}).

\subsection{\spp{Gekko}}
Posterior probability weakly supports no shared divergences
(Figure~\ref{fig:gekkonevents}), but Bayes factors weakly seven divergence event
across the eight pairs (Figure~\ref{fig:gekkonevents}), suggesting a shared
divergence between
\spp{G.\ mindorensis} on the Islands of Panay and Masbate
and
\spp{G.\ porosus} on the Islands of Sabtang and Batan
(Figure~\ref{fig:gekkodivtimes}).
Under the intermediate prior on the concentration parameter, the support
increases for seven events and a shared divergence between
\spp{G.\ mindorensis} on the Islands of Panay and Masbate
and
\spp{G.\ porosus} on the Islands of Sabtang and Batan
(Figures S\ref{fig:gekkoneventsbyconcentration} \& S\ref{fig:gekkodivtimesbyconcentration}).
Under the prior that puts most of the probability on one shared event,
there the posterior probability prefers six divergences
(Figures S\ref{fig:gekkoneventsbyconcentration})
with another shared divergence between
\spp{G.\ mindorensis} on the Islands of Babuyan Claro and Calayan
and
\spp{G.\ porosus} on the Islands of Romblon and Tablas
(Figures S\ref{fig:gekkodivtimesbyconcentration}), however,
Bayes factors still prefer seven divergences.

Similarly, as the prior on divergence times becomes more diffuse,
the results shift from ambiguity between seven or eight divergence
events, to ambiguity between six or seven events, to strong
support for six events, with the same island pairs sharing
divergences
(Figures S\ref{fig:gekkoneventsbytimeprior} \& S\ref{fig:gekkodivtimesbytimeprior}).

As with \spp{Cyrtodactylus}, the estimates of divergence times
are robust to whether polyallelic sites are recoded or removed
(Figure S\ref{fig:gekkodivtimesbytimeprior}),
and population size estimates are robust to 
priors on \concentration and \divtime, and whether polyallelic sites
are recoded or removed
(Figures S\ref{fig:gekkopopsizesbyconcentration} \& S\ref{fig:gekkopopsizesbytimeprior}).

The Islands of Babuyan Claro and Calayan were never connected,
so the support for the shared divergence between
\spp{G.\ mindorensis} on the Islands of Babuyan Claro and Calayan
and
\spp{G.\ porosus} on the Islands of Romblon and Tablas
are likely an artifact under the most extreme priors on 
\concentration and \divtime 
(Figures S\ref{fig:gekkodivtimesbyconcentration} \& S\ref{fig:gekkodivtimesbytimeprior}).
However, the weak support for a shared divergence between
\spp{G.\ mindorensis} on the Islands of Panay and Masbate
and
\spp{G.\ porosus} on the Islands of Sabtang and Batan
under more reasonable priors is interesting because
both pairs of islands were potentially connected during
glacial cycles.

Under the priors we initially chose as appropriate (as opposed to those used to
assess prior sensitivity), the posterior probability that the Panay-Masbate and
Sabtang-Batan pairs co-diverged is 0.385.
We could calculate a Bayes factor by using the fact that under the Dirichlet
process and given the concentration parameter, the prior probability that any
two pairs share the same divergence time is $\frac{1}{1 + \concentration}$.
However, this would not be appropriate, because we did not identify
the Panay-Masbate and
Sabtang-Batan pairs of interest
\emph{a priori}, but rather were drawn to these pairs based
on the posterior sample.
Thus, the probability that \emph{any} two pairs share the same divergence
is no longer the appropriate prior probability for our Bayes factor calculation.
Rather, we need to consider the prior probability that the two pairs with
the most similar divergence times share the same divergence.
To get this prior probability, wee can take advantage of the fact that this
condition is met anytime the number of divergence events is less than eight.
Thus, the prior probability that the two pairs with the most similar divergence
times share the same divergence is equal to one minus the prior probability
that all eight pairs diverge independently.
Under our prior on the concentration parameter of $\distgamma(1.1, 56.1)$,
this prior probability is approximately 0.5.
Thus, our posterior probability for the co-divergence between the Panay-Masbate
and Sabtang-Batan pairs actually \emph{less} than the prior probability,
resulting in a weak Bayes factor of approximately 1.6 in support \emph{against}
the co-divergence.
Thus, as Bayesians, we should favor the explanation that this co-divergence is
due chance.

Also, it's interesting to note that in analyses of both genera we see
support for shared divergences increase as the prior on divergence times
becomes more diffuse.
This is the same pattern seen in approximate-likelihood Bayesian approaches to
this problem \citep{Oaks2012,Hickerson2013,Oaks2014reply}.
Hickerson et al.\ \citeyear{Hickerson2013} proposed this pattern was
called by numerical problems, whereas Oaks et al.\ \citeyear{Oaks2014reply}
found support for the problem being more fundamental:
as more prior density is placed in regions of divergence-time space where the
likelihoods tend to be low, models that have fewer divergence-time parameters
have greater marginal likelihoods, because their likelihood is ``averaged''
over less space with low likelihood and substantive prior weight.
Our results clearly support the latter explanation, as the
MCMC approach used here did not suffer from the insufficient posterior sampling
proposed by Hickerson et al.\ \citeyear{Hickerson2013}.

\section{Discussion}
