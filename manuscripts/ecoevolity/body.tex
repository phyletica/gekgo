\section{Introduction}

An open question in biogeography is to what extent does the landscape, and
changes to it, predict diversification.

Hypothesis: Repeated fragmentation of islands caused vicariant diversification.
This predicts that divergences between populations on separate islands that
were fragmented were temporally clustered and corresponded with interglacial
rises in sea level.

We finally have the data and method for a powerful evaluation of this
hypothesis.


\section{Methods}

\subsection{Sampling}
For two genera of geckos, \spp{Cyrtodactylus} and \spp{Gekko}, we sampled
individuals from pairs of populations that occur on two different islands.
Because the climate-mediated fragmentation of the islands was a relatively
recent phenomenon, we selected pairs of populations that were inferred to be
closely related from previous genetic data \citep{Siler2012, Siler2014kikuchii,
    Welton2010, Welton2010zootaxa, Siler2010}.
Our goal was to avoid deeply divergent pairs that we know \emph{a priori} could
not have been isolated during Late Pliocene and Pleistocene glacial cycles.
We also sought to sample pairs that span islands that were connected during
glacial periods, as well as islands that were never connected.
We included the latter as ``controls;'' because these islands were never
connected, the divergence between the populations inhabiting them must be
dispersal mediated, the timing of which is not expected to be clustered across
pairs.
Applying these criteria, we ended up with eight pairs of populations for
\spp{Cyrtodactylus}, four of which, span islands that were connected during
glacial periods (Table \ref{table:comparisons} \& \highLight{SX}).
For \spp{Gekko} we also sampled eight pairs of populations, one of which span
previously connected islands (Table \ref{table:comparisons} \& \highLight{SX}).
\thought{Is it worth including \spp{Gekko}? With only one ``PAIC'' pair, it
    cannot address the hypothesis of interest.}

\subsection{Genomic library preparation and sequencing}

We extracted DNA from tissue using the guanidine thiocyanate method described
by Esselstyn et.\ al.\ \citeyear{Esselstyn2008}.
We diluted the extracted DNA for each individual to a concentration of 5
ng/$\mu$L based on the initial concentration measured with a Qubit 2.0
Fluorometer.
We generated three restriction-site associated DNA sequence (RADseq) libraries,
each with 96 individuals, using the multiplexed shotgun genotyping (MSG)
protocol of Andolfatto et.\ al.\ \citep{Andolfatto2011}.
Following digestion of 50 ng of DNA with the NdeI restriction enzyme, we
ligated each sample to one of 96 adaptors with a unique six base-pair barcode.
After pooling the 96 samples together, we selected 250--300bp fragments to
remain in the library using a Pippen Prep.
For each pool of 96 size-selected samples, we performed eight separate
polymerase chain reactions for 14 cycles (PCR) using Phusion High-Fidelity PCR
Master Mix (NEB Biolabs) and primers that bind to common regions in the
adaptors.
Following PCR, we did two rounds of AMPure XP bead cleanup (Beckman Coulter,
Inc) using a 0.8 bead volume to sample ratio.
Each library was sequenced in one lane of an Illumina Hiseq 2500 high-output
run, with single-end 100bp reads.
We provide information on all of the individuals included in our three RADseq
libraries in \highLight{Table~SX}, a subset of which were included in the
population pairs we analyzed for this study (Table \ref{table:comparisons} \&
\highLight{SX}).

\subsection{Data assembly}
We used ipyrad version 0.7.13 \citep{ipyrad0713} to assemble the raw RADseq
reads into loci.
To maximize the number of assembled loci, we de novo assembled the reads
separately for each pair of populations.
\highLight{Details on ipyrad settings \ldots}

\subsection{Inferring shared divergences}
We approach the inference of temporally clustered divergences as a problem of
model choice.
Our goal is to treat the number of divergence events shared (or not) among the
pairs of populations, and the assignment of the pairs to those events, as
random variables to be estimated from the aligned sequence data.
For eight pairs, there are 4,140 possible divergence models.
I.e., there are 4,140 ways to partition the eight pairs to $\nevents = 1, 2,
\ldots, 8$ divergence events \citep{Bell1934,Oaks2014dpp,Oaks2018ecoevolity}.
Whereas divergences caused by sea-level rise would not happen simultaneously,
we expect that on a timescale of the lizards' mutation rate, treating them as
simultaneous should be a better explanation of the data than treating them as
completely independent.

Given the large number of models, and our goal of making probability statements
about them, we used a Bayesian model-averaging approach.
Specifically, we used a Dirichlet process prior on the assignment of our
pairs to an unknown number of divergence times.
We did this using the full-likelihood Bayesian framework implemented in the
software package \ecoevolity \citep{Oaks2018ecoevolity}.
The Dirichlet process is specified by a
(1) concentration parameter, \concentration, which determines how probable it
is for pairs to share the same divergence event, \emph{a priori}, and
(2) base distribution, which serves as the prior on the divergence-time
parameters.

Based on previous data \citep{Welton2010, Welton2010zootaxa, Siler2010} we
assumed a prior of \dexponential{0.005} on divergence times for our eight pairs
of \spp{Cyrtodactylus} populations, in units of substitutions per site.
To explore the sensitivity of our results to this assumption, we also
tried a prior of \dexponential{0.05} on the divergence times.
We also assumed a hyperprior of $\distgamma(1.5, 3.13)$ on the concentration
parameter of the Dirichlet process, which corresponds to a prior mean number of
divergence events of five.
To explore the sensitiviy of our results to this assumption, we also
tried a hyperprior of
$\distgamma(1.1, 56.1)$
and
$\distgamma(0.5, 1.31)$,
which place 50\% of the prior probability on the
$\nevents = 8$
and
$\nevents = 1$
models, respectively.

Based on previous data \citep{Siler2012, Siler2014kikuchii}, we assumed a prior
of \dexponential{0.0005} on divergence times for our eight pairs of \spp{Gekko}
populations, in units of substitutions per site.
To explore the sensitivity of our results to this assumption, we also tried
priors of \dexponential{0.005}, \dexponential{0.05} on the \spp{Gekko}
divergence times.
For the concentration parameter of the Dirichlet process, we explored the same
three hyperpriors described above.

For all analyses of both the \spp{Cyrtodactylus} and \spp{Gekko} data, we
assumed equal mutation rates among the pairs, a prior distribution of
\dgamma{4.0}{0.004} on the effective effective size of the populations, and a
prior distribution of \dgamma{100}{1} on the relative effective size of the
ancestral population (relative to the mean size of the two descendant
populations).
The model implemented in \ecoevolity assumes each character is unlinked (i.e.,
evolved along a gene tree that is independent conditional on the population
tree).
However, by analyzing simulated data, Oaks \citeyear{Oaks2018ecoevolity} showed
the method performs better when all linked sites are used than when data are
excluded to avoid violating the assumption of unlinked sites.
Accordingly, we analyzed all of the sites of our RADseq loci.
The model implemented in \ecoevolity is also restricted to characters with two
possible states (hereafter referred to as polyallelic sites).
Thus, for sites with three or more nucleotides, we compared how sensitive our results were to two different strategies:
(1) removing polyallelic sites, and
(2) recoding the sites as biallelic by coding each state as either having the
first nucleotide in the alignment or a different nucleotide.

For each analysis, we ran 10 independent MCMC chains for 150,000 generations,
sampling every 100th generation.
We assessed convergence and mixing of the chains by inspecting the potential scale
reduction factor (the square root of Equation 1.1 in Brooks and Gelman
\citeyear{Brooks1998}) and effective sample size \citep{Gong2014} of the
likelihood and all parameters using the sumchains tool of \pycoevolity.
We also visually inspected the trace of the log likelihood and parameters over
generations with the program Tracer version 1.6 \citep{Tracer16}.


\section{Results}

Figure~\ref{fig:cyrtdivtimes}

Figure~\ref{fig:gekkodivtimes}

Figure~\ref{fig:cyrtnevents}

Figure~\ref{fig:gekkonevents}

Figure~\ref{fig:cyrtpopsizes}

Figure~\ref{fig:gekkopopsizes}

\section{Discussion}
